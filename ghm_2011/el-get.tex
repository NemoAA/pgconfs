\documentclass{beamer}

\usepackage{beamerthemesplit}
\usetheme{Warsaw}

\title{el-get}
\subtitle{Emacs distributed packaging}
\author{Dimitri Fontaine}
\date{August 26, 2011}

\begin{document}

\frame{\titlepage}

\section*{Outline}
\frame{
  \frametitle{Table of contents}
  \tableofcontents
}

\section{el-get}

\subsection{el what?}

\frame{
  \frametitle{Emacs Lisp Extensions}

  Installing extensions for emacs is uneasy, unclean.

  \begin{itemize}
   \item<1-> Search EmacsWiki, another source, find \textit{something}
   \item<1-> Copy it (\texttt{wget}) somewhere in \texttt{load-path}
   \item<1-> If not a single file, \texttt{(add-to-list 'load-path ...)}
   \item<2-> Read further install notes or docs
   \item<3-> Test it, use it, adapt your setup to it
   \item<4-> Remove it?
   \item<5-> Upgrade... where did I get it from already?
  \end{itemize}
}

\frame{
  \frametitle{apt-get for Emacs}

  \texttt{el-get} offers \texttt{apt-get} like \textit{UI} for Emacs.

  \begin{itemize}
   \item \texttt{M-x el-get-install}
   \item \texttt{M-x el-get-describe}
   \item \texttt{M-x el-get-list-packages}
   \item \texttt{M-x el-get-self-update}
   \item \texttt{M-x el-get-update}
  \end{itemize}
}

\frame{
  \frametitle{ELPA}

  The Emacs Lisp Packages Archive is a good solution, that is:

  \begin{itemize}
   \item Integrated in Emacs24
   \item Supports non-GNU repositories (yes, it's a feature)
   \item Often needs editing the \texttt{elisp} source
   \item Only simple for one-file packages
   \item Strange with respect to versioning
  \end{itemize}
}

\subsection{Distributed, what do you mean?}

\frame{
  \frametitle{Recipes}

  el-get wants your recipes!

  \begin{itemize}
   \item<1-> recipes are packaging script
   \item<2-> can be pretty simple
   \item<2-> can be pretty involved
   \item<3-> easy enough to make your own locally
   \item<3-> \texttt{el-get-sources} or in a local file
   \item<4-> require no edits of upstream sources
  \end{itemize}
}

\begin{frame}[fragile]
  \frametitle{Recipe Example 1, el-get}

  \texttt{el-get} is self-hosted and has a \texttt{*scratch*} installer

  \begin{verbatim}
(:name el-get
       :website "https://github.com/dimitri/el-get#readme"
       :description "Manage the external elisp bits and pieces you depend upon."
       :type git
       :branch "2.stable"
       :url "https://github.com/dimitri/el-get.git"
       :features el-get
       :load    "el-get.el"
       :compile "el-get.el")
  \end{verbatim}
\end{frame}

\begin{frame}[fragile]
  \frametitle{Recipe Example 2, escreen}

  Escreen is available on some website, not ELPA ready, Copyright 1992, 94,
  95, 97, 2001, 2005 Noah S. Friedman

  \begin{verbatim}
(:name escreen
       :description "Emacs window session manager"
       :type http
       :url "http://www.splode.com/~friedman/software/emacs-lisp/src/escreen.el"
       :post-init (lambda ()
		    (autoload 'escreen-install "escreen" nil t)))
  \end{verbatim}
\end{frame}

\begin{frame}[fragile]
  \frametitle{Recipe Example 3}

  Solarized is a recent color-theme, and has been a fast moving target,
  hosted at github, not available as an ELPA package.

  \begin{verbatim}
(:name color-theme-solarized
       :description "Emacs highlighting using Ethan Schoonover's Solarized color scheme"
       :type git
       :url "https://github.com/sellout/emacs-color-theme-solarized.git"
       :load "color-theme-solarized.el"
       :depends color-theme)
  \end{verbatim}
\end{frame}

\frame{
  \frametitle{Sharing Recipes}

  Distributed packaging: the Emacs Packages Social Network

  \begin{itemize}
   \item Once tested, send them to me (github, mail)
   \item Once pushed, \texttt{el-get-self-update}
   \item Available in \texttt{M-x el-get-install}
   \item and in \texttt{M-x el-get-describe}
   \item and in \texttt{M-x el-get-list-packages}
  \end{itemize}
}

\frame{
  \frametitle{Sharing Recipes}

  Distributed packaging: the Emacs Packages Social Network
  \linebreak
  \linebreak
  \pause

  Or just mail them over to a friend.  Save as \texttt{recipe-name.el}
  somewhere in \texttt{el-get-recipe-path}.
}

\subsection{Installing el-get}

\begin{frame}[fragile]
  \frametitle{The \texttt{*scratch*} installer}

  el-get aims at making installing things like el-get easier.  How to
  bootstap?

  \begin{verbatim}
;; So the idea is that you copy/paste this code into your *scratch* buffer,
;; hit C-j, and you have a working el-get.
(url-retrieve
 "https://github.com/dimitri/el-get/raw/master/el-get-install.el"
 (lambda (s)
   (end-of-buffer)
   (eval-print-last-sexp)))
  \end{verbatim}

\end{frame}

\begin{frame}[fragile]
  \frametitle{\texttt{.emacs}}

  In your \texttt{~/.emacs.d/init.el} (well, \texttt{user-init-file}):

  \begin{verbatim}
(setq el-get-sources
      '((:name magit
	     :after (lambda ()
             (global-set-key (kbd "C-x C-z") 'magit-status)))))

(setq my-packages
      (append
       '(cssh el-get switch-window mailq escreen emms)
       (mapcar 'el-get-source-name el-get-sources)))

(el-get 'sync my-packages)
  \end{verbatim}
\end{frame}

\section{Features}

\subsection{Package Types Supported}

\frame{
  \frametitle{git, http, svn, what else?}

  \begin{itemize}
   \item<1-> git (with branch support), http, svn
   \item<2-> emacswiki, \texttt{M-x el-get-emacswiki-refresh}
   \item<3-> emacsmirror
   \item<4-> git-svn, bzr, darcs, cvs, hg
   \item<4-> ftp, http-tar
   \item<4-> apt-get, pacman, fink
   \item<4-> elpa (multi repositories, \texttt{:repo})
  \end{itemize}
}

\subsection{el-get-init}

\frame{
  \frametitle{what does el-get cares about for me?}

  When a packaged is installed, then it's initialized, and again at next
  startups.  Meaning:

  \begin{itemize}
   \item<1-> \texttt{load-path}
   \item<1-> \texttt{Info-directory-list} and \textt{ginstall-info}
   \item<2-> byte compiling (\texttt{el-get-byte-compile-at-init})
   \item<3-> autoloads
   \item<3-> \texttt{require}, \texttt{load}, \texttt{eval-after-load}
   \item<4-> custom function (prepare, before, post-init, after)
   \item<4-> el-get-post-init-hooks
  \end{itemize}
}

\frame{
  \frametitle{Sync or Async? Eager or Lazy?}

  You can be lazy, it works.  Or you can cook an involved init sequence so
  that there's no \textit{eager} loading of anything, so that Emacs starts
  in less than a second.  \pause Once a month.

  \begin{itemize}
   \item<3-> \texttt{autoloads}, \texttt{el-get-is-lazy}
   \item<3-> \texttt{:after}, \texttt{:features}
   \item<4-> \texttt{(el-get 'sync)}
   \item<4-> \texttt{(el-get 'wait)}
  \end{itemize}
}

\subsection{Other Features}

\frame{
  \frametitle{What else...}

  \begin{itemize}
   \item<1-> Windows support (users called) (yes they still exist)
   \item<1-> Package dependencies
   \item<1-> Per \texttt{system-type} build commands
   \item<2-> 72 contributors (including some GNU Hackers)
   \item<2-> 402 recipes included
   \item<2-> 1656 emacswiki files supported
   \item<2-> 3054 emacsmirror repositories
   \item<2-> 3627 lines of emacs-lisp fun (including comments)
   \item<3-> 1-year old, long time stable, production ready
  \end{itemize}
}

\end{document}
