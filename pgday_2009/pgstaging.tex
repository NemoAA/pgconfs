\documentclass{beamer}
\usepackage[latin9]{inputenc}
\usepackage[T1]{fontenc}

\usepackage{beamerthemesplit}
\usetheme{Warsaw}
\beamertemplatetransparentcovered

\title{pg\_staging}
\author{Dimitri Fontaine}
\date{November, 7th 2009}

\begin{document}

\frame{\titlepage}

\section*{Agenda}
\frame{
  \frametitle{How to manage your staging environments?}
  \tableofcontents
}

\section{Premise: you do have backups, right?}

\frame{
  \frametitle{About backups}

  The idea of \texttt{pg\_staging} is to maintain a \textit{staging}
  environment from production backups. If you don't have one, this tool will
  not do any good for you, but you have bigger problems than that. Supported
  backups are:

  \begin{itemize}
   \item<2-> \texttt{pg\_dump}
   \item<3-> \textit{PITR archives} (ongoing work)
  \end{itemize}
}

\section{Describing staging envs}

\frame{
  \frametitle{About backups}

  When developping \texttt{pg\_staging}, the aim is to manage a staging
  environment with:

  \begin{itemize}
   \item<2-> a dev env, with new code and old database containing
     developpers test data
   \item<3-> a prelive env with code to get in production and the most
     recent possible data from production
  \end{itemize}
}

\section{pg\_staging design \& dependancies}
\subsection{Design}

\frame{
  \frametitle{Design}

  What do we want the tool to do?

  \begin{itemize}
   \item<2-> easily restore production database on staging env
   \item<3-> filtering out (\textit{historical}) data we don't need in staging
   \item<4-> allow to restore more than one version of production database
   \item<5-> allow to easily switch from one version to the other
   \item<6-> offer interactive console usage and be cron friendly too
  \end{itemize}
}

\begin{frame}[fragile]
  \frametitle{Easy restore with filtering}

  The \texttt{restore} command will \texttt{createdb}, \textit{fetch} the
  wanted backup, \textit{filter} the dump catalog, \texttt{pg\_restore}
  selected data then optionaly switch the staging env to this new database.
  \linebreak
  \pause

  \begin{example}
  \begin{verbatim}
schemas         = public, payment, utils, jdb, operations, statistiques
schemas_nodata  = sessions, archives
  \end{verbatim}
  \end{example}
\end{frame}

\frame{
  \frametitle{Dependancies}

  The following tools are used by \texttt{pg\_staging}:

  \begin{itemize}
   \item<2-> \texttt{apache} to serve the backups
   \item<3-> \texttt{pgbouncer} for database switching
   \item<4-> \texttt{postgresql-client-8.x} for dump \& restore
   \item<5-> \texttt{staging-client.sh}
   \item<6-> non-interactive \texttt{ssh}
   \item<7-> \texttt{python}
  \end{itemize}
}

\subsection{pgbouncer}

\begin{frame}[fragile]
  \frametitle{Switching databases with pgbouncer}

  \texttt{pgbouncer} is able to provide ``virtual'' database:
  \pause

  \begin{example}
  \begin{verbatim}
pg_staging> pgbouncer some_db.dev
              some_db      some_db_20091029 :5432
     some_db_20090717      some_db_20090717 :5432
     some_db_20091029      some_db_20091029 :5432
  \end{verbatim}
  \end{example}

  \pause

  So \texttt{pg\_staging} is able to switch staging database without editing
  application connection strings.
\end{frame}
}

\subsection{pg\_restore}

\frame{
  \frametitle{Filtering objects from dump files}

  Using \texttt{pg\_restore} options -l and -L, \texttt{pg\_staging} can filter
  out objects at restore time, per schema.
  \linebreak 
  \linebreak

  A 2-pass parsing is done, so that even triggers depending on functions
  that are in the given excluded schemas are filtered out.
  \linebreak
  \linebreak
  \pause

  See commands \texttt{catalog} \& \texttt{triggers} to check for yourself
  the catalog that will get used.
}

\begin{frame}[fragile]
  \frametitle{Cluster globals, init command}

  To be able to restore you need to \textit{create} the roles. Store next to
  your backups \texttt{pg\_dumpall --globals-only} and edit your setup
  accordingly.

  \pause
  
  \begin{example}
  \begin{verbatim}
dumpall_url = /clusters/myserver/8.3-main.globals.sql
  \end{verbatim}
  \end{example}

\end{frame}

\begin{frame}[fragile]
  \frametitle{custom \texttt{SQL}}

  Setup a \texttt{sql\_path} entry in your configuration file then \texttt{restore} command will play all files in there:

  \pause

  \begin{example}
  \begin{verbatim}
psql ... -f pre/*.sql
pg_restore
psql ... -f post/*.sql
  \end{verbatim}
  \end{example}
\end{frame}

\subsection{pg\_dump}

\frame{
  \frametitle{\texttt{pg\_dump} support}

  \texttt{pg\_staging} also provides a \texttt{dump} command for producing a
  local dump from the remote configured server.
  \linebreak
  \linebreak
  \pause

  Use \texttt{redump} if you want to overwrite a previous existing dump file.  
}

\subsection{londiste}

\frame{
  \frametitle{londiste support}

  Given a \textit{replication.ini} setup (to be documented soon, but we
  actually use the feature), \texttt{pg\_staging} is able to configure
  \textit{londiste} and the associated \textit{pgq ticker}.
  \linebreak
  \linebreak
  \pause

  It will also skip restoring tables we are a subscriber of, see
  \texttt{nodata} command.
}

\begin{frame}[fragile]
  \frametitle{replication.ini example}

\begin{overprint}

\onslide<1>
\begin{verbatim}
[DEFAULT]
pgq_lazy_fetch  = 500

[some_db_ticker.dev]
job_name = pgq_some_db
host     = bdd1.service.dev
db       = dbname=some_db user=postgres port=5432 host=127.0.0.1
\end{verbatim}

\onslide<2>
\begin{verbatim}
[payment_to_reporting.dev]
host            = bdd2.service.dev
ticker          = some_db_ticker.dev
pgq_queue_name  = service_to_reporting
provider        = some_service.dev
subscriber      = some_reporting.dev
provider_db     = dbname=some_service host=localhost port=6432 user=postgres
subscriber_db   = dbname=some_reporting host=reporting.service.dev port=6432 user=postgres
provides = schema.table1 schema.table2 other_schema.other_table
\end{verbatim}
\end{overprint}
\end{frame}

\section{usage and documentation}
\subsection{generics}

\frame{
  \frametitle{pg\_staging notions}

  Some generics about \texttt{pg\_staging} usage:

  \begin{itemize}
   \item<2-> \textit{commands} are the same in console and \textrt{CLI}
   \item<3-> \textit{config} is made of \texttt{.INI} files
   \item<4-> \textit{dbname} refers to .INI section
   \item<5-> \textit{backup date} defaults to \textit{today}
  \end{itemize}
}

\subsection{commands overview}
\begin{frame}[fragile]
  \frametitle{main commands}

\begin{verbatim}
pg_staging> help
  commands provide a user friendly listing of commands
      init <dbname>
   restore <dbname> restore a database
      drop <dbname> drop given database
      dump <dbname> dump a database
    redump dump a database, overwriting the pre-existing dump file if it exists
      pitr <dbname> <time|xid> value
\end{verbatim}
\end{frame}

\begin{frame}[fragile]
  \frametitle{listing resources}

\begin{verbatim}
pg_staging> help
       ...
 databases list configured databases
   backups list available backups for a given database
    dbsize show given database size
   dbsizes show dbsize for all databases of a dbname section
      psql launch a psql connection to the given configured section
\end{verbatim}
\end{frame}

\begin{frame}[fragile]
  \frametitle{and some more}

\begin{verbatim}
pg_staging> help
       ...
     fetch <dbname> [date]
      load <dbname> <dumpfile>

      show show given database setting current value
       get <dbname> <option> print the current value of [dbname] option
       set <dbname> <option> <value> for current session only
\end{verbatim}
\end{frame}

\begin{frame}[fragile]
  \frametitle{managing services}
\begin{verbatim}
pg_staging> help
       ...
 pgbouncer list configured pgbouncer databases
     pause pause <dbname> [date] (when no date given, not expanded to today)
    resume resume <dbname> [date] (when no date given, not expanded to today)
    switch <dbname> <bdate> switch default pgbouncer config to dbname_bdate

  londiste prepare londiste files for providers of given dbname section
    status show status of given service ...
     start start given service depending on its configuration and special args
      stop stop given service depending on its configuration and special args
   restart restart given service depending on its configuration and special args
\end{verbatim}
\end{frame}

\begin{frame}[fragile]
  \frametitle{internals you might have a use for}
\begin{verbatim}
pg_staging> help
       ...
   catalog <dbname> [dump] print catalog for dbname, edited for nodata tables
  triggers <dbname> [dump] print triggers procedures for dbname
    nodata list tables to restore without their data
    presql <dbname> [date]
   postsql <dbname> [date]
search_path alter database <dbname> set search_path
\end{verbatim}
\end{frame}

\subsection{Interactive console, CLI, scripting}
\begin{frame}[fragile]
  \frametitle{Interactive or not?}

  Remember the ``design'' slide, we want to have both an interactive tool
  and a script friendly (think \texttt{cron}) one.

  \pause

  \begin{example}
  \begin{verbatim}
#  pg_staging restore mydb today
#  pg_staging < foo.pgs
#  pg_staging
Welcome to pg_staging 0.7.
pg_staging> 
  \end{verbatim}
  \end{example}
\end{frame}

\section{distribution and status}
\subsection{development \& releases}

\begin{frame}[fragile]
  \frametitle{pgfoundry or github...}

  \url{http://pgfoundry.org/projects/pgstaging} will host the releases when
  they happen.

  \pause

  \begin{example}
  \begin{verbatim}
# pg_staging.py --version
pg_staging 0.7
  \end{verbatim}
  \end{example}

  \pause

  \url{http://github.com/dimitri/pg_staging} is hosting the code, along with
  the debian packaging. Go \texttt{git clone} and try it, we use it about
  daily.
\end{frame}

\subsection{The end. Any question?}
\begin{frame}[fragile]
  \frametitle{Any question?}

  \center{Now is the time to ask!}

  \linebreak
  \pause

  \begin{center}
  If you want to leave feedback, consider visiting
  \url{http://2009.pgday.eu/feedback}
  \end{center}
\end{frame}

\end{document}
