\documentclass{beamer}

\usepackage{beamerthemesplit}
\usetheme{Warsaw}

\title{PgLoader, the parallel ETL for PostgreSQL}
\author{Dimitri Fontaine}
\date{\today}

\begin{document}

\frame{\titlepage}

\section*{Outline}
\frame{
  \frametitle{Table of contents}
  \tableofcontents
}

\section{Introduction}
\subsection{pgloader, the what?}

\frame
{
  \frametitle{ETL}

  \begin{definition}
    An \alert{ETL} process data to load into the database from a flat
    file.
  \end{definition}

  \begin{itemize}
   \item<1-> Extract
   \item<2-> Transform
   \item<3-> Load
  \end{itemize}
}

\frame
{
  \frametitle{pgloader's features}

  \begin{itemize}
   \item<1-> Load CSV data
   \item<2-> Load pretend-to-be CSV data
   \item<3-> Continue loading when confronted to errors
   \item<4-> Apply user define transformation to data, on the fly
   \item<5-> Optionaly have all your cores participate into processing
  \end{itemize}
}

\section{Architecture}
\subsection{Main components}

\frame{
  \frametitle{Configuration}

  We first parse the configuration, with templating system
}

\frame{
  \frametitle{Loading: file reading}

  \begin{itemize}
   \item<1-> Read files one line at a time
   \item<2-> Parse physical lines into logical lines
   \item<3-> Supports several readers
    \begin{itemize}
     \item<4-> textreader
     \item<3-> csvreader
     \item<5-> fixedreader
    \end{itemize}
  \end{itemize}
}

\frame{
  \frametitle{Processing lines}

  \begin{itemize}
   \item<1-> columns restrictions
   \item<2-> columns reordering
   \item<3-> user defined columns (constants)
   \item<4-> user defined reformating modules
  \end{itemize}
}

\frame{
  \frametitle{COPYing to PostgreSQL}

  \begin{itemize}
   \item<1-> cStringIO buffers
   \item<2-> configurable size (copy\_every)
   \item<3-> using copy\_expert() when available (CVS)
   \item<4-> dichotomic error search
  \end{itemize}
}

\frame{
  \frametitle{Handling of erroneous data input}

  \begin{itemize}
   \item<1-> reject data file
   \item<2-> 
  \end{itemize}
}

\frame{
  \frametitle{Error logging}

  \begin{itemize}
   \item<1-> log file
   \item<2-> console log level: \texttt{client\_min\_messages}
   \item<3-> logfile log level: \texttt{log\_min\_messages}
  \end{itemize}
}

\subsection{Parallel Organisation}

\frame{
  \frametitle{Why going parallel?}

  Loading is IO bound, not CPU bound, right?

  \begin{itemize}
   \item <2-> for large disks array, \textit{not so much}
   \item <3-> with complex parsing, \textit{not so much}
   \item <4-> with heavy user rewritting, \textit{not so much}
  \end{itemize}
}

\begin{frame}[fragile]
  \frametitle{Ok... How?}

  \begin{itemize}
   \item <1-> mutli-threading is easy to start with in python
   \item <2-> then you add in dequeues and semaphores (critical sections) and signals
   \item <3-> Giant Interpreter Lock
   \item <4-> fork() based reimplementation could be of interrest
  \end{itemize}

  \begin{example}
  \begin{verbatim}
    class PGLoader(threading.Thread):
  \end{verbatim}
  \end{example}

\end{frame}

\frame{
  \frametitle{Parallelism choices}

  Has beed asked by some \-hackers, their use cases dictated two different modes.

  The idea is to have a parallel pg\_restore testbed, interresting with large input
  files (100GB to several TB). Perfs on on core can't be as good as server one, due
  to client\-server roundtrips compared to local file reading, but with some more CPUs
  feeding the disk array, should show up nice improvements.

  Testing and feebback more than welcome\!
}

\frame{
  \frametitle{Round robin reader}

  \begin{example}
   split\_file\_reading = False
  \end{example}
}


\frame{
  \frametitle{Split file reader}

  \begin{example}
   split\_file\_reading = True
  \end{example}
}

\section{Configuration file format}
\subsection{INI files with user friendly parsing}

\frame{
  \frametitle{How to set configuration knobs}

   \begin{itemize}
    \item <1-> brackets sections
    \item <2-> \texttt{variable = value}
    \item <3-> templates
   \end{itemize}
}

\frame{
   \frametitle{User friendly and flexible}

   \begin{itemize}
    \item <1-> Error messages (try to be nice to users)
    \item <2-> Flexible input (only\_cols \= 1\-3, 5)
   \end{itemize}
}

\subsection{User level extensions}

\begin{frame}[fragile]
  \frametitle{User defined columns}

  Constant columns added at parsing time.

  Use case: adding an origin\_server\_id field depending on the file
  to get loaded, for data aggregation.

  \begin{example}
  \begin{verbatim}
   columns        = b:2, d:1, x:3, y:4
   udc_c          = constant value
   copy_columns   = b, c, d
  \end{verbatim}
  \end{example}
\end{frame}

\frame
{
  \frametitle{User defined Reformating modules}

  The basic idea is not to use any sed-like preprocessing scripts.
}

\frame
{
  \frametitle{User defined Reformating modules}

}

\subsection{Configuration Examples}

\begin{frame}[fragile]
  \frametitle{simple}

  \begin{example}
  \begin{verbatim}
[simple]
table        = simple
filename     = simple/simple.data
format       = text
datestyle    = dmy
field_sep    = |
trailing_sep = True
columns      = *
  \end{verbatim}
  \end{example}
\end{frame}

\frame{
  \frametitle{others}

  csv, fixed, user defined columns, columns reordering, user defined modules, parralel loading
  cluttered data
}

\section{Usage}

\section{Incoming features}

\frame{
  \frametitle{TODO}

  TODO is published at http://pgloader.projects.postgresql.org/dev/TODO.html

  \begin{itemize}
   \item <1-> Constraint Exclusion support
   \item <2-> Reject Behaviour
   \item <3-> XML support with user defined XSLT StyleSheet
   \item <4-> Facilities
  \end{itemize}

  Don't be shy and just ask for new features!
}

\section{Project management}

\frame{
  \frametitle{Resources and Users}

  pgfoundry, 1 developper, some users, no mailing list yet, some mails sometime

  \#postgresql and \#postgresqlfr

  packages for debian, freebsd, openbsd
}

\end{document}
