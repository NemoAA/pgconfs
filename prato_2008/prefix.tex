\documentclass{beamer}

\usepackage{beamerthemesplit}
\usetheme{Warsaw}

\title{The prefix PostgreSQL module and custom GiST indexing}
\author{Dimitri Fontaine}
\date{\today}

\begin{document}

\frame{\titlepage}

\section*{Outline}
\frame{
  \frametitle{Table of contents}
  \tableofcontents
}

\section{Introduction}
\subsection{Once upon a time}

\frame{
  \frametitle{prefix queries}

  The \alert{prefix} project is about solving prefix queries where the literal is not
  the literal but the column data.

  \begin{example}
    SELECT ... FROM prefixes WHERE prefix @> 'abcdef';
  \end{example}

  You want to find rows where prefix is 'a', 'abc', 'abcd', etc.
}

\subsection{Available solutions}

\begin{frame}[fragile]
  \frametitle{SQL or procedural approches}
  
  depesz (both active on the IRC \texttt{\#postgresql} channel and the lists, has a blog entry
  about it: http://www.depesz.com/index.php/2008/03/04/searching-for-longest-prefix/
  
  It requires creating a lot of specific indexes
  
  \begin{example}
  \begin{verbatim}
     CREATE INDEX pa3_1 on prefixes_andrewsn_3 (prefix) WHERE length(prefix) = 1;
     CREATE INDEX pa3_2 on prefixes_andrewsn_3 (prefix) WHERE length(prefix) = 2;
     CREATE INDEX pa3_3 on prefixes_andrewsn_3 (substring(prefix for 3)) WHERE length(prefix) >= 3;
  \end{verbatim}
  \end{example}
\end{frame}

\section{First version, text indexing}

\frame{
  \frametitle{Developping a GiST indexing module}
}

\section{Optimisation: prefix ranges}

\frame{
  \frametitle{A new \textit{varlena} datatype}
}

\section{Going "live"}

\frame{
  \frametitle{Current CVS version, 0.3, is running "live"}
}

\section{Next versions}

\begin{frame}[fragile]
  \frametitle{Some more optimisation}
  
  \texttt{prefix} next version will provide some more optimisation
  by having its internal data structure accept wider ranges of prefixes.
  The user visible part of this will the the input format of the \texttt{prefix\_range}
  datatype:
  
  \begin{example}
  \begin{verbatim}
    SELECT 'abc[def-xyz]'::prefix_range;
  \end{verbatim}
  \end{example}
\end{frame}

\section{project organisation}

\frame{
  \frametitle{Yet Another Single Man Project}
  
  \texttt{prefix} project is using pgfoundry hosting facilities, has no mailing-list
  and currently only onde maintainer, contributions and usage feedbacks are
  welcome.
}

\end{document}
