\documentclass{beamer}

\usepackage{beamerthemesplit}
\usetheme{Warsaw}

\title{The prefix PostgreSQL module and custom GiST indexing}
\author{Dimitri Fontaine}
\date{October 18, 2008}

\begin{document}

\frame{\titlepage}

\section*{Outline}
\frame{
  \frametitle{Table of contents}
  \tableofcontents
}

\section{Introduction: problem and existing solutions}

\begin{frame}[fragile]
  \frametitle{prefix queries}

  The \alert{prefix} project is about solving prefix queries where the
  literal is not the literal but the column data.

  \begin{example}
  \begin{verbatim}
    SELECT ... FROM prefixes WHERE prefix @> 'abcdef';
  \end{verbatim}
  \end{example}

  You want to find rows where prefix is 'a', 'abc', 'abcd', etc.
\end{frame}

\begin{frame}[fragile]
  \frametitle{The plain SQL way}
  
  \texttt{depesz} has a blog entry about it:
  \url{http://www.depesz.com/index.php/2008/03/04/searching-for-longest-prefix/}
  
  \begin{example}
  \begin{verbatim}
    create table prefixes (
        id serial primary key,
        prefix text not null unique,
        operator text,
        something1 text,
        something2 text
    );
  \end{verbatim}
  \end{example}
\end{frame}
  
\begin{frame}[fragile]
  \frametitle{The plain SQL way: indexes for known length 3}

  This works well when you know about the prefix length in your queries:

  \begin{example}
  \begin{verbatim}
     CREATE INDEX pa1 on prefixes (prefix) 
            WHERE length(prefix) = 1;

     CREATE INDEX pa2 on prefixes (prefix)
            WHERE length(prefix) = 2;

     CREATE INDEX pa3 on prefixes (substring(prefix for 3))
            WHERE length(prefix) >= 3;
  \end{verbatim}
  \end{example}
\end{frame}


\begin{frame}[fragile]
  \frametitle{The plain SQL way: indexes for known length 3}

  This works well when you know about the prefix length in your queries:

  \begin{example}
  \begin{verbatim}
     select * from prefixes
     where ( length(prefix) = 1 and prefix = ? )
        or ( length(prefix) = 2 and prefix = ? )
        or ( length(prefix) >= 3 
             and substring(prefix for 3) = ? )
  order by length(prefix) desc
  limit 1;
  \end{verbatim}
  \end{example}
\end{frame}

\begin{frame}[fragile]
  \frametitle{The plain SQL way: no extra indices}

   \texttt{depesz} thought of simply using a list of generated prefixes of
   phone number. For example for phone number \texttt{0123456789}, we would
   have: \texttt{prefix in ('0', '01', '012', '0123', ...)}.

  \begin{example}
  \begin{verbatim}
     select *
     from prefixes
     where prefix in (?, ?, ?, ?, ?, ?, ?)
     order by length(prefix) desc
     limit 1;
  \end{verbatim}
  \end{example}
\end{frame}

\begin{frame}[fragile]
  \frametitle{The GiST index way}

  The generic solution here is the specialized \alert{GiST} index.

  \begin{example}
  \begin{verbatim}
    CREATE INDEX idx_prefix ON prefixes
           USING GIST(prefix gist_prefix_ops);

    SELECT ... FROM prefixes WHERE prefix @> 'abcdef';
  \end{verbatim}
  \end{example}

  So let's talk about developing this solution!
\end{frame}


\section{Developing a GiST module}

\frame{
  \frametitle{What's GiST?}

  A kind of index for PostgreSQL: Generalized Search Tree.

\begin{columns}[c]

\pause
\column{.5\textwidth}
  PostgreSQL supports several kind of indexes:

  \begin{itemize}
   \item<3-> BTree
   \item<4-> Hash
   \item<5-> GiST 
   \item<6-> GIN
  \end{itemize}

\pause
\column{.5\textwidth} 
  \linebreak
  \linebreak
  \onslide<7-> What's special about \texttt{GiST}?

  \begin{itemize}
   \item<8-> balanced index
   \item<9-> tree-structured access method
   \item<10-> acts as a base template
   \linebreak
   \linebreak
  \end{itemize}
  
\end{columns}

  \onslide<11-> It's a kind of a \textit{plug-in} index system, easy enough
  to work with to plug your own datatype smartness into PostgreSQL index
  searches.

}

\subsection{PostgreSQL module development}

\frame{
  \frametitle{Developing a GiST indexing module}

  Big picture steps:

  \begin{itemize}
   \item<1-> internal representation of data
   \item<2-> a \textit{standard} PostgreSQL extension module
   \item<3-> exporting C functions in SQL
   \item<4-> using \texttt{pgxs}
  \end{itemize}
}


\begin{frame}[fragile]
  \frametitle{\texttt{prefix\_range} datatype}

  Internal representation of data is the following:

  \begin{example}
  \begin{verbatim}
typedef struct {
  char first;
  char last;
  char prefix[1]; /* varlena struct, data follows */
} prefix_range;
  \end{verbatim}
  \end{example}

  It came from internal representation to full new SQL visible datatype,
  \texttt{prefix\_range}.
\end{frame}

\begin{frame}[fragile]
  \frametitle{PostgreSQL module development}

  This part of the development is the same whether you're targeting index
  code or general purpose code. It's a rather a steep learning
  curve... You'll have to read the source.

  Helpers: \url{http://doxygen.postgresql.org/} and \texttt{\#postgresql}

  \begin{example}
  \begin{verbatim}
DatumGetCString(
 DirectFunctionCall1(
  prefix_range_out,
  PrefixRangeGetDatum(orig)
 )
)
  \end{verbatim}
  \end{example}
\end{frame}

\begin{frame}[fragile]
  \frametitle{PostgreSQL module development: multi-version support}

  If you want to support multiple major versions of PostgreSQL, check
  \texttt{PG\_VERSION\_NUM} and... read the source to find out about
  discrepancies.

  \begin{example}
  \begin{verbatim}
#if PG_VERSION_NUM / 100 == 802
#define PREFIX_VARSIZE(x)        (VARSIZE(x) - VARHDRSZ)
#define PREFIX_VARDATA(x)        (VARDATA(x))

#if PG_VERSION_NUM / 100 == 803
#define PREFIX_VARSIZE(x)        (VARSIZE_ANY_EXHDR(x))
#define PREFIX_VARDATA(x)        (VARDATA_ANY(x))
  \end{verbatim}
  \end{example}
\end{frame}

\begin{frame}[fragile]
  \frametitle{PostgreSQL module development: macros}

  PostgreSQL code style uses macros to simplify raw C-structure accesses,
  the extension modules are better using the same technique.

  \begin{example}
  \begin{verbatim}
#define DatumGetPrefixRange(X)	      ((prefix_range *) PREFIX_VARDATA(DatumGetPointer(X)) )
#define PrefixRangeGetDatum(X)	      PointerGetDatum(make_varlena(X))
#define PG_GETARG_PREFIX_RANGE_P(n)  DatumGetPrefixRange(PG_DETOAST_DATUM(PG_GETARG_DATUM(n)))
#define PG_RETURN_PREFIX_RANGE_P(x)  return PrefixRangeGetDatum(x)
  \end{verbatim}
  \end{example}
\end{frame}


\begin{frame}[fragile]
  \frametitle{PostgreSQL module development: function declarations}

  PostgreSQL has support for polymorphic and overloading functions, even at
  its innermost foundation: C-level code.

  \begin{example}
  \begin{verbatim}
PG_FUNCTION_INFO_V1(prefix_range_cast_from_text);
Datum prefix_range_cast_from_text(PG_FUNCTION_ARGS)
{
  text *txt = PG_GETARG_TEXT_P(0);
  Datum cstring = DirectFunctionCall1(textout, 
                                      PointerGetDatum(txt));
  return DirectFunctionCall1(prefix_range_in, cstring);
}
  \end{verbatim}
  \end{example}
\end{frame}

\begin{frame}[fragile]
  \frametitle{PostgreSQL module development: \texttt{SQL} integration}

  Here's how to declare previous function in \texttt{SQL}:

  \begin{example}
  \begin{verbatim}
CREATE OR REPLACE FUNCTION prefix_range(text)
RETURNS prefix_range
AS 'MODULE_PATHNAME', 'prefix_range_cast_from_text'
LANGUAGE 'C' IMMUTABLE STRICT;

CREATE CAST (text as prefix_range)
  WITH FUNCTION prefix_range(text) AS IMPLICIT;
  \end{verbatim}
  \end{example}
\end{frame}

\frame{
  \frametitle{PostgreSQL module development: allocating memory}

  \begin{itemize}

   \item<1-> Use \texttt{palloc} unless told not to, or when the code you're
     getting inspiration from avoids \texttt{palloc} for \texttt{malloc}.

   \item<2-> \texttt{palloc} memory lives in a \textit{Context} which is
     freed in one sweep at its death (end of query execution, end of
     transaction, etc).

   \item<3-> PostgreSQL has support for polymorphic and overloading
     functions, even at the C-level.
  \end{itemize}
}

\begin{frame}[fragile]
  \frametitle{PostgreSQL module development: building with \texttt{pgxs}}

  PostgreSQL provides the tool suite for easy building and integration of
  your module: put the following into a \texttt{Makefile}

  \begin{example}
  \begin{verbatim}
MODULES = prefix
DATA_built = prefix.sql

PGXS = $(shell pg_config --pgxs)
include $(PGXS)
  \end{verbatim}
  \end{example}
\end{frame}

\begin{frame}[fragile]
  \frametitle{PostgreSQL module development: configuring}

  When developing a PostgreSQL extension, you'll find it convenient for
  your installation to exports \texttt{DEBUG} symbols and check for C-level
  \texttt{Assert}s.

  \begin{example}
  \begin{verbatim}
./configure --prefix=/home/dim/pgsql  \
            --enable-debug            \
            --enable-cassert
  \end{verbatim}
  \end{example}
\end{frame}

\begin{frame}[fragile]
  \frametitle{New datatype magic}

  We choose to export the internal data structure as a full type:

  \begin{example}
  \begin{overprint}

    \onslide<1>
  \begin{verbatim}
CREATE TYPE prefix_range (
        INPUT   = prefix_range_in,
        OUTPUT  = prefix_range_out,
        RECEIVE = prefix_range_recv,
        SEND    = prefix_range_send
);
  \end{verbatim}

    \onslide<2>
  \begin{verbatim}
dim=# select '0123'::prefix_range | '0137' as union;
  union
---------
 01[2-3]
(1 row)
  \end{verbatim}

    \onslide<3->
  \begin{verbatim}
CREATE TABLE prefixes (
       prefix    prefix_range primary key,
       name      text not null,
       shortname text,
       state     char default 'S',
);
  \end{verbatim}

  \end{overprint}
  \end{example}
  
  \onslide<4> SQL integration means column storage too! \alert{wow}.
\end{frame}

\subsection{GiST specifics}

\begin{frame}[fragile]
  \frametitle{The \texttt{GiST} interface API}

  To code a new \texttt{GiST} index, one only has to code 7 functions in a
  dynamic module for PostgreSQL:

\begin{columns}[c]

\column{.5\textwidth}
  \begin{itemize}
   \item<1,2,6-> \texttt{consistent()}
   \item<1,3,6-> \texttt{union()}
   \item<1,4,6-> \texttt{compress()}
   \item<1,4,6-> \texttt{decompress()}
   \item<1,5,6-> \texttt{penalty()}
   \item<1,5,6-> \texttt{picksplit()}
   \item<1,2,6-> \texttt{same()}
  \end{itemize}

\column{.5\textwidth}

  \begin{overprint}

  \onslide<2>

  All entries in a \textit{subtree} will share any property you
  implement.
  \linebreak
  \texttt{StrategyNumber} is the operator used into the query.
  \linebreak
  \linebreak
  You get to implement your equality operator (\texttt{=},
  \texttt{pr\_eq}) for the internal datatype in the index.

  \onslide<3>

  Input: a set of entries. 
  \linebreak
  \linebreak
  Output: a new data which is \textit{consistent} with all of them.
  \linebreak
  \linebreak
  This will form the index tree non-leaf elements, any element in a subtree
  has to be consistent with all the nodes atop.

  \onslide<4> 
  Index internal leaf data.
  \begin{example}
  \begin{verbatim}
PG_FUNCTION_INFO_V1(gpr_compress);
Datum gpr_compress
 (PG_FUNCTION_ARGS) 
{  PG_RETURN_POINTER(
     PG_GETARG_POINTER(0));
}
  \end{verbatim}
  \end{example}

  \onslide<5>
  In order for your \texttt{GiST} index to show up good performance
  characteritics, you'll have to take extra care in implementing good
  versions of those two.
  \linebreak
  \linebreak
  \textit{see next slides}

  \onslide<6>

  Those functions expects \textit{internal} datatypes as argument and return
  values, and stores \textit{exactly} this.
  \linebreak
  \linebreak
  It's easy to mess it up and have \texttt{CREATE INDEX}
  \texttt{segfault}. \alert{Assert()} your code.

  \end{overprint}

\end{columns}
\end{frame}

\begin{frame}[fragile]
  \frametitle{\texttt{GiST} SQL integration: \texttt{opclass}}

  You declare \texttt{OPERATOR CLASS}es over the datatype to tell PostgreSQL
  how to index your data. It's all dynamic down to the datatypes, operator
  and indexing support. Another \textit{wow}.

  \pause

  \begin{example}
  \begin{verbatim}
CREATE OPERATOR CLASS gist_prefix_range_ops
FOR TYPE prefix_range USING gist 
AS
    OPERATOR  1  @>,
    FUNCTION  1  gpr_consistent (internal, prefix_range, prefix_range)
    ...
  \end{verbatim}
  \end{example}
\end{frame}


\subsection{GiST challenges}

\begin{frame}[fragile]
  \frametitle{GiST penalty}

  Is this user data more like this other one or that other one?

  \begin{example}
  \begin{overprint}

  \onslide<1>
  \begin{verbatim}
select a, b, pr_penalty(a::prefix_range, b::prefix_range)
  from
      
      
      
      
      
      
      
order by 3 asc;
  \end{verbatim}

  \onslide<2>
  \begin{verbatim}
select a, b, pr_penalty(a::prefix_range, b::prefix_range)
  from (values('095[4-5]', '0[8-9]'),
              ('095[4-5]', '0[0-9]'),
              ('095[4-5]', '[0-3]'),
              ('095[4-5]', '0'),
              ('095[4-5]', '[0-9]'),
              ('095[4-5]', '0[1-5]'),
              ('095[4-5]', '32'),
              ('095[4-5]', '[1-3]')) as t(a, b)
order by 3 asc;
  \end{verbatim}

  \onslide<3>
  \begin{verbatim}
select a, b, pr_penalty(a::prefix_range, b::prefix_range)
  from (values





              ('095[4-5]', '32'),
              ('095[4-5]', '[1-3]')) as t(a, b)
order by 3 asc;
  \end{verbatim}

  \onslide<4>
  \begin{verbatim}
select a, b, pr_penalty(a::prefix_range, b::prefix_range)
  from (values('095[4-5]', '0[8-9]'),
              ('095[4-5]', '0[0-9]'),





                                   ) as t(a, b)
order by 3 asc;
  \end{verbatim}

  \onslide<5>
  \begin{verbatim}
select a, b, pr_penalty(a::prefix_range, b::prefix_range)
  from (values
              
              ('095[4-5]', '[0-3]'),
              ('095[4-5]', '0'),
              ('095[4-5]', '[0-9]'),
              ('095[4-5]', '0[1-5]'),
              
                                   ) as t(a, b)
order by 3 asc;
  \end{verbatim}

  \onslide<6>
  \begin{verbatim}
    a     |   b    | gpr_penalty
----------+--------+-------------
 095[4-5] | 0[8-9] | 1.52588e-05
 095[4-5] | 0[0-9] | 1.52588e-05
 095[4-5] | [0-3]  |  0.00390625
 095[4-5] | 0      |  0.00390625
 095[4-5] | [0-9]  |  0.00390625
 095[4-5] | 0[1-5] |   0.0078125
 095[4-5] | 32     |           1
 095[4-5] | [1-3]  |           1
  \end{verbatim}

  \end{overprint}
  \end{example}

\end{frame}

\begin{frame}[fragile]
  \frametitle{GiST picksplit}

  The index grows as you insert data, remember?

  \begin{overprint}

  \onslide<2>
  ~
  \linebreak
  prefix picksplit first pass step: presort the \texttt{GistEntryVector}
  vector by positionning the elements sharing the non-empty prefix which is
  the more frequent in the distribution at the beginning of the vector.
  \linebreak
  \linebreak

  \onslide<3>
  ~
  \linebreak
  prefix picksplit first pass step: presort the \texttt{GistEntryVector}
  vector by positionning the elements sharing the non-empty prefix which is
  the more frequent in the distribution at the beginning of the vector.
  \linebreak
  \linebreak
  Then consume the vector by both ends, compare them and choose to move them
  in the \textit{left} or ther \textit{right} side of the new subtree.

  \onslide<4>
  \begin{example}
  \begin{verbatim}
Datum pr_picksplit(GistEntryVector *entryvec,
                   GIST_SPLITVEC *v,
                   bool presort)
{
    OffsetNumber maxoff = entryvec->n - 1;
    GISTENTRY *ent      = entryvec->vector;

    nbytes = (maxoff + 1) * sizeof(OffsetNumber);
  \end{verbatim}
  \end{example}

  \onslide<5>
  \begin{example}
  \begin{verbatim}
    listL  = (OffsetNumber *) palloc(nbytes);
    listR  = (OffsetNumber *) palloc(nbytes);

    unionL = DatumGetPrefixRange(ent[offl].key);
    unionR = DatumGetPrefixRange(ent[offr].key);
  \end{verbatim}
  \end{example}

  \onslide<6>
  \begin{example}
  \begin{verbatim}
    pll = __pr_penalty(unionL, curl);
    plr = __pr_penalty(unionR, curl);
    prl = __pr_penalty(unionL, curr);
    prr = __pr_penalty(unionR, curr);
  \end{verbatim}
  \end{example}

  \onslide<7>
  \begin{example}
  \begin{verbatim}
    if( pll <= plr && prl >= prr )       { l, r }
    else if( pll > plr && prl >= prr )   {  , r }
    else if( pll <= plr && prl < prr )   { l,   }
    else if( (pll - plr) < (prr - prl) ) { all to l }
    else { /* all to listR */ }
  \end{verbatim}
  \end{example}

  \end{overprint}

\end{frame}

\subsection{Testing, debugging, tools}

\begin{frame}[fragile]
  \frametitle{dataset}

  ART is the French Telecom Regulation Authority, provides a list of all
  prefixes for local operators. Let's load some \texttt{11966} prefixes from
  \url{http://www.art-telecom.fr/fileadmin/wopnum.rtf} .

  \pause

  \begin{example}
  \begin{verbatim}
dim=# select prefix, shortname from prefixes limit 5;                                                                                                              
  prefix  | shortname 
----------+-----------
 010001[] | COLT
 010002[] | EQFR
 010003[] | NURC
 010004[] | PROS
 010005[] | ITNF
(5 rows)
  \end{verbatim}
  \end{example}

\end{frame}

\begin{frame}[fragile]
  \frametitle{gevel}

  The \texttt{gevel} module allows to SQL query any \texttt{GiST} index!

  \begin{example}
  \begin{overprint}

  \onslide<2>
  \begin{verbatim}
dim=# select gist_stat('idx_prefix');
 Number of levels:          2
 Number of pages:           63
 Number of leaf pages:      62
 Number of tuples:          10031
 Number of invalid tuples:  0
 Number of leaf tuples:     9969
 Total size of tuples:      279904 bytes
 Total size of leaf tuples: 278424 bytes
 Total size of index:       516096 bytes
  \end{verbatim}

  \onslide<3>
  \begin{verbatim}
select *
  from gist_print('idx_prefix')
    as t(level int, valid bool, a prefix_range)
where level =1;

select *
  from gist_print('idx_prefix')
    as t(level int, valid bool, a prefix_range)
order by level;
  \end{verbatim}

  \end{overprint}
  \end{example}
\end{frame}

\begin{frame}[fragile]
  \frametitle{Correctness testing}

  Even when your index builds without a \texttt{segfault} you have to
  test. It can happen at query time, or worse:
  \pause

  \begin{example}
  \begin{verbatim}
set enable_seqscan to on;
select * from prefixes where prefix @> '0146640123';
select * from prefixes where prefix @> '0100091234';

set enable_seqscan to off;
select * from prefixes where prefix @> '0146640123';
select * from prefixes where prefix @> '0100091234';
  \end{verbatim}
  \end{example}
\end{frame}


\begin{frame}[fragile]
  \frametitle{Performance testing}

  \begin{example}

  \begin{overprint}

  \onslide<1>
  \begin{verbatim}
create table numbers(number text primary key);
insert into numbers
  select '01' || to_char((random()*100)::int, 'FM09')
              || to_char((random()*100)::int, 'FM09')
              || to_char((random()*100)::int, 'FM09')
              || to_char((random()*100)::int, 'FM09')
   from generate_series(1, 5000);
INSERT 0 5000
  \end{verbatim}

  \onslide<2>
  \begin{verbatim}
dim=# explain analyze 
      SELECT * 
        FROM numbers n 
             JOIN prefixes r 
               ON r.prefix @> n.number;
  \end{verbatim}

  \onslide<3>
  \begin{verbatim}
 Nested Loop
  (cost=0.00..4868614.00 rows=149575000 width=45)
  (actual time=0.345..4994.296 rows=10213 loops=1)
   ->  Seq Scan on numbers n
        (cost=0.00..375.00 rows=25000 width=11)
        (actual time=0.015..12.917 rows=25000 loops=1)
   ->  Index Scan using idx_prefix on ranges r
        (cost=0.00..104.98 rows=5983 width=34)
        (actual time=0.182..0.197 rows=0 loops=25000)
         Index Cond: (r.prefix @> (n.number)::prefix_range)
 Total runtime: 4998.936 ms
(5 rows)
  \end{verbatim}

  \end{overprint}
  \end{example}
\end{frame}


\section{Current status and roadmap}

\begin{frame}[fragile]
  \frametitle{Status \& Roadmap}

  \begin{itemize}

  \item<1-> Current release is \texttt{0.3-1} and \texttt{CVS} version is
    live!

     \textit{and has been involved into more than 7 million calls, 2 lookups per
       call}

   \item<2-> Open item \#1: add support for indexing \texttt{text} data
     directly, using \texttt{prefix\_range} internally without the user
     noticing.

   \item<3-> Open item \#2: implement a simple optimisation idea (see next
     slide).

   \item<4-> Release Version 1.0, go into maintenance mode!

  \end{itemize}
\end{frame}

\begin{frame}[fragile]
  \frametitle{Some more optimisation}
  
  \texttt{prefix} next version will provide some more optimisation by having
  its internal data structure accept wider ranges of prefixes.  The user
  visible part of this will the the input format of the
  \texttt{prefix\_range} datatype:

  \pause

  \begin{example}
  \begin{verbatim}
    SELECT 'abc[def-xyz]'::prefix_range;
  \end{verbatim}
  \end{example}
\end{frame}

\frame{
  \frametitle{Project Organisation}
  
  \texttt{prefix} project is using \url{http://pgfoundry.org} hosting
  facilities, has no mailing-list and currently one maintainer,
  contributions and usage feedbacks are more than welcome.
  \linebreak

  \pause

   While developing the solution, the IRC channel \texttt{\#postgresql} was
   a great resource, especially thanks to the invaluable help from
   \texttt{RhodiumToad} formely known as \texttt{AndrewSN}, \alert{Andrew
     Gierth}.

}

\end{document}
