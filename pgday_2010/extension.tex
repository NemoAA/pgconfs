%%% PgDay 2010
%%%
%%% PostgreSQL extensibility is remarkable but incomplete. It lacks dump and
%%% restore support. What that means is  that once an extension is installed
%%% into your database, PostgreSQL currently has no idea of what SQL objects
%%% belongs to  the extension rather  itself, so  the dump will  contain the
%%% instructions to install the extension. That's only practical if you want
%%% to restore  your dump targeting  the very same extension's  version, but
%%% when upgrading systems that's seldom what happens. This talk will detail
%%% how to fix this problem and more, explaining you how to benefit from the
%%% extensions capabilities for your own work within the database.

\documentclass[english]{beamer}
\usepackage[utf8,latin9]{inputenc}
%%\usepackage[latin9]{inputenc}
\usepackage[T1]{fontenc}
\usepackage{babel}

\usepackage{beamerthemesplit}
\usetheme{Warsaw}
\beamertemplatetransparentcovered

\title{PostgreSQL extension's development}
\author{Dimitri Fontaine}
\date{Dec. 7, 2010}

\begin{document}

\frame{\titlepage}

\section*{Agenda}
\frame{
  \frametitle{Content}
  \tableofcontents[pausesections]
}

\section{What's an Extensions?}

\begin{frame}[fragile]
  \frametitle{Definitions}

  \begin{center}
    PostgreSQL extensibility is remarkable but incomplete. 
  \end{center}

\begin{example}[Basic SQL query]
\begin{verbatim}
  SELECT col
    FROM table
   WHERE stamped > date 'today' - interval '1 day'
\end{verbatim}
\end{example}
\end{frame}

\begin{frame}[fragile]
  \frametitle{Some extensions example}

  \begin{center}
    46 Contribs, Community extensions, Private ones...
  \end{center}

\begin{columns}[c]

\column{.18\textwidth}
  \begin{itemize}
   \item cube
   \item ltree
   \item citext
   \item \alert{hstore}
   \item intagg
  \end{itemize}

\column{.23\textwidth}
  \begin{itemize}
   \item adminpack
   \item \alert{pgq}
   \item pg\_trgm
   \item wildspeed
   \item \alert{dblink}
  \end{itemize}

\column{.23\textwidth} 
  \begin{itemize}
   \item \alert{PostGIS}
   \item ip4r
   \item temporal
   \item prefix
   \item pgfincore
  \end{itemize}

\column{.4\textwidth}
  \begin{itemize}
   \item pgcrypto
   \item pg\_stattuple
   \item pg\_freespacemap
   \item pg\_stat\_statements
   \item \alert{pg\_standby}
  \end{itemize}

\end{columns}
\end{frame}

\frame{
  \begin{center}
    PostgreSQL extensibility is remarkable but incomplete. 
    \linebreak

    \alert{It lacks dump and restore support.}
  \end{center}
}

\subsection{Current state of affairs}

\begin{frame}[fragile]
  \frametitle{Installing an extension}

  \begin{example}[Installing an extension before 9.1]
\begin{verbatim}
apt-get install postgresql-contrib-9.0
apt-get install postgresql-9.0-ip4r
psql -f /usr/share/postgresql/9.0/contrib/hstore.sql
\end{verbatim}
  \end{example}

  \begin{itemize}
    \item<2-> so, what did it install? ok, reading the \textit{script}
    \item<3-> Oh, nice, it's all in the \texttt{public} schema
    \item<4-> Oh, very nice, no \texttt{ALTER OPERATOR SET SCHEMA}
  \end{itemize}

  \onslide<5>
  \begin{center}
    Wait, it gets better!
  \end{center}
\end{frame}

\frame{
  \frametitle{backup and restores}

  \begin{center}
    \texttt{pg\_dump -h remote mydb | psql fresh}\
  \end{center}

  \begin{itemize}
    \item<1-> extensions objects are an entire part of your database
    \item<1-> but they are maintained elsewhere, that's just a dependency
    \item<2-> \texttt{pg\_dump} makes no difference
    \item<3-> what about upgrading systems (system, database, extension)
  \end{itemize}  
}

\section{The extension specs \& scope}

\subsection{scope}
\frame{
  \frametitle{What problems are we solving?}

  Clearing up the mess. No feature is accepted in PostgreSQL without
  complete support for dump and restore nowadays. And that's good news.
}

\subsection{specs}

\frame{
  \frametitle{How are we solving our problems?}

  Tracking dependencies, relying on the system's packaging
}

\subsection{Implementation details}

\frame{
  \frametitle{The effort in figures}

  \begin{center}
    \texttt{git diff master.. | diffstat | tail -1}
    \linebreak
    \texttt{244 files changed, 4940 insertions(+), 1362 deletions(-)}
  \end{center}

  \begin{itemize}
    \item<2-> 5 patches, 6 branches, its own \textit{Commit Fest} section
    \item<3-> about 18 months to get an agreement on what to develop \textit{first}
    \item<4-> 2 \textit{Developer Meeting} interventions, in Ottawa, \textit{PgCon}
    \item<5-> 4 weeks full time, countless evening
  \end{itemize}
}

\frame{
  \frametitle{What's to know, now}

  \begin{itemize}
    \item \texttt{CREATE EXTENSION hstore WITH SCHEMA utils;}
    \item \texttt{\char`\\dx[+]}
    \item \texttt{select * from pg\_extension\_objects('unaccent');}
    \item \texttt{ALTER EXTENSION hstore SET SCHEMA addons;}
    \item \texttt{DROP EXTENSION hstore CASCADE;}
    \item \texttt{pg\_execute\_sql\_file()}
    \item \texttt{pg\_execute\_sql\_string()}
  \end{itemize}

}
\section{Extension for their authors. That's YOU. Here.}

\subsection{PGXS and the control file}

\begin{frame}[fragile]
  \frametitle{Using PGXS}

  Simpler way to have your files installed at the right place, using
  \texttt{make install}. But \texttt{Makefile}s are hard, right?

  \onslide<2>
  \begin{example}[citext.control.in]
\begin{verbatim}
MODULES = citext
DATA_built = citext.sql
DATA = uninstall_citext.sql
REGRESS = citext

EXTENSION = $(MODULE_big)
EXTVERSION = $(VERSION)
\end{verbatim}
  \end{example}
\end{frame}

%% FIXME: PGXS example Makefile, prefix
%% FIXME: PGXS example Makefile, SQL only

\begin{frame}[fragile]
  \frametitle{The control file}

  It's a very complex file containing the \textit{meta data} that PostgreSQL
  needs to know about to be able to register your \textit{extension} in its
  \textit{system catalogs}. It looks like this:

  \onslide<2>
  \begin{example}[citext.control.in]
\begin{verbatim}
# citext
comment = 'case-insensitive character string type'
version = 'EXTVERSION'
\end{verbatim}
  \end{example}
\end{frame}

\frame{
  \frametitle{custom\_variable\_classes}

  FIXME
}

\subsection{extension with user data}

\frame{
  \frametitle{User data}

  The problem: \textit{extension} support sole goal is not to pollute
  \texttt{pg\_dump} output. And now you're putting \textit{user} data in
  there.
}

\begin{frame}[fragile]
  \frametitle{User data and extension's scripts}

  \begin{example}[user extension's script]
\begin{verbatim}
DO $$
BEGIN
IF pg_extension_with_user_data() THEN
  create schema foo;
  create table foo.bar(id serial primary key);
  perform pg_extension_flag_dump('foo.bar_id_seq'::regclass);
  perform pg_extension_flag_dump('foo.bar::regclass);
END IF;
END;
$$;
\end{verbatim}
  \end{example}
\end{frame}

\subsection{extension and packaging}

\frame{
  \frametitle{debian and pg\_buildext}

  \texttt{postgresql-server-dev-all}

  %%% FIXME: include some example
}

\section{Conclusion}

\subsection{Sponsoring}

\frame{
  \frametitle{Money}

  4 week full time at home, thanks to \alert{2ndQuadrant}, and to our
  affiliation with European Research

  \begin{quote}
    The research leading to these results has received funding from the
    European Union's Seventh Framework Programme (FP7/2007-2013) under grant
    agreement 258862
  \end{quote}
}

\subsection{Any question?}

\begin{frame}[fragile]
  \frametitle{Any question?}

  \begin{center}
    Now is a pretty good time to ask!
  \end{center}

  \linebreak
  \begin{center}
    If you want to leave feedback, consider visiting
    \url{http://2010.pgday.eu/feedback}
  \end{center}
\end{frame}

\end{document}
