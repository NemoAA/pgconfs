%%% Skytools3 --- what's good and new, why you want to upgrade, how to plan
%%% your upgrade

\documentclass[english]{beamer}
\usepackage[utf8,latin9]{inputenc}
%%\usepackage[latin9]{inputenc}
\usepackage[T1]{fontenc}
\usepackage{babel}

\usepackage{beamerthemesplit}
\usetheme{Warsaw}
\beamertemplatetransparentcovered

\title{Skytools 3.0}
\author{Dimitri Fontaine}
\date{July, 12 2011}

\begin{document}

\frame{\titlepage}

\section*{Agenda}
\frame{
  \frametitle{Content}
  \tableofcontents[pausesections]
}

\section{Skytools}

\subsection{Software Architecture}

\frame{
  \frametitle{Skytools Architecture}

  \begin{itemize}
   \item in-database \texttt{PGQ} modules
   \item Skytools python framework
   \item PGQ python librairies, utilities to code consumers (daemons)
   \item Specific consumers, such as \texttt{londiste}
   \item Extra tools based upon the framework, such as \texttt{walmgr}
  \end{itemize}
}

\section{Skytools 3.0}

\subsection{PGQ improvements}

\frame{
  \frametitle{PGQ improvements}

  %%% TODO
}

\subsection{Replicating Queues}

\frame{
  \frametitle{PGQ Nodes}

  PGQ architecture has changed so that the queue themselves can be
  replicated from a node to another, to allow for online switchover
  facilities.  That comes with 3 node types:

  \begin{itemize}
   \item Root
   \item Branch
   \item Leaf
  \end{itemize}  
}

%%% TODO, add a text graph with root/branch/leaf example, straight from
%%% qadmin

\subsection{Cooperative Consumers}

\frame{
  \frametitle{Cooperative Consumers}

  PGQ is often used on its own, for general queuing needs that are not tied
  to replication.  In some cases you want to have multiple consumers
  draining in parallel from the same queue, but \textit{sharing} the work.
  \linebreak
  \pause
  
  Includes failover.  When a consumer stops consuming, its current batch
  gets reassigned to another one after a configurable timeout.
}

\subsection{Admin tool, \tettt{qadmin}}

\frame{
  \frametitle{\texttt{qadmin}}

  \texttt{qadmin} is the new interactive command line, replacing
  \texttt{pgqadm} and offering a much richer set of commands.

  \begin{itemize}
   \item connect, install
   \item create queue, alter queue, show ...
   \item register, unregister
   \item londiste commands
  \end{itemize}  
}

\subsection{Londiste improvements}

\frame{
  \frametitle{Londiste improvements}

  Londiste itself received some new features

  \begin{itemize}
   \item Parallel \texttt{COPY}
   \item Execute scripts (\texttt{DDL} support)
   \item Online topology changes
   \item \texttt{londiste ... add --create}
   \item Custom handlers (\textit{plugins})
   \item Documentation...
  \end{itemize}
}

\subsection{\texttt{debian} Packaging}

\frame{
  \frametitle{\texttt{debian} Packaging}

  Completely new for \texttt{skytools3}

  \begin{itemize}
   \item skytools3
   \item python-pgq3
   \item python-skytools3
   \item skytools3-walmgr
   \item skytools3-ticker
   \item postgresql-8.4-pgq3
   \item postgresql-9.0-pgq3
  \end{itemize}
}

\section{Upgrading}

\subsection{Upgrading PGQ}

\frame{
  \frametitle{Upgrading PGQ}

  %%% TODO
}

\subsection{Upgrading Services}

\frame{
  \frametitle{Upgrading Services}

  That's the easy part

  \begin{itemize}
   \item Tickers: \texttt{pgqd} takes over \texttt{pgqadm}
   \item Daemons: the API is compatible
   \item Mostly: no more \texttt{failed} queue
   \item Londiste should run fine
  \end{itemize}  
}

\end{document}

\section{Conclusion}

\subsection{Any question?}

\frame{
  \frametitle{Any question?}

  \begin{center}
    Now is a pretty good time to ask!
  \end{center}
}

\end{document}
