%%% PGCON 2011
%%%
%%% PostgreSQL extensibility is remarkable but incomplete. It lacks dump and
%%% restore support. What that means is  that once an extension is installed
%%% into your database, PostgreSQL currently has no idea of what SQL objects
%%% belongs to  the extension rather  itself, so  the dump will  contain the
%%% instructions to install the extension. That's only practical if you want
%%% to restore  your dump targeting  the very same extension's  version, but
%%% when upgrading systems that's seldom what happens. This talk will detail
%%% how to fix this problem and more, explaining you how to benefit from the
%%% extensions capabilities for your own work within the database.

\documentclass[english]{beamer}
\usepackage[utf8,latin9]{inputenc}
%%\usepackage[latin9]{inputenc}
\usepackage[T1]{fontenc}
\usepackage{babel}

\usepackage{beamerthemesplit}
\usetheme{Warsaw}
\beamertemplatetransparentcovered

\title{PostgreSQL extension's development}
\author{Dimitri Fontaine}
\date{May 20, 2011}

\begin{document}

\frame{\titlepage}

\section*{Agenda}
\frame{
  \frametitle{Content}
  \tableofcontents[pausesections]
}

\section{What's an Extension?}

\frame{
  \begin{center}
    What is a ``Extension''?
  \end{center}
}

\begin{frame}[fragile]
  \frametitle{Definitions}

  \begin{center}
    PostgreSQL extensibility is remarkable but incomplete. 
  \end{center}

\begin{example}[Basic SQL query]
\begin{verbatim}
  SELECT col
    FROM table
   WHERE stamped > date 'today' - interval '1 day'
\end{verbatim}
\end{example}
\end{frame}

\begin{frame}[fragile]
  \frametitle{Some extensions example}

  \begin{center}
    46 Contribs, Community extensions, Private ones...
  \end{center}

\begin{columns}[c]

\column{.18\textwidth}
  \begin{itemize}
   \item cube
   \item ltree
   \item citext
   \item \alert{hstore}
   \item intagg
  \end{itemize}

\column{.23\textwidth}
  \begin{itemize}
   \item adminpack
   \item \alert{pgq}
   \item pg\_trgm
   \item wildspeed
   \item \alert{dblink}
  \end{itemize}

\column{.23\textwidth} 
  \begin{itemize}
   \item \alert{PostGIS}
   \item ip4r
   \item temporal
   \item prefix
   \item pgfincore
  \end{itemize}

\column{.4\textwidth}
  \begin{itemize}
   \item pgcrypto
   \item pg\_stattuple
   \item pg\_freespacemap
   \item pg\_stat\_statements
   \item \alert{pg\_standby}
  \end{itemize}

\end{columns}
\end{frame}

\frame{
  \begin{center}
    PostgreSQL extensibility is remarkable but incomplete. 
    \linebreak

    \alert{It lacks dump and restore support.}
  \end{center}
}

\subsection{Before 9.1 and CREATE EXTENSION}

\frame{
  \begin{center}
    Before 9.1 and CREATE EXTENSION
  \end{center}
}

\begin{frame}[fragile]
  \frametitle{Installing an extension}

  \begin{example}[Installing an extension before 9.1]
\begin{verbatim}
apt-get install postgresql-contrib-9.0
apt-get install postgresql-9.0-ip4r
psql -f /usr/share/postgresql/9.0/contrib/hstore.sql
\end{verbatim}
  \end{example}

  \begin{itemize}
    \item<2-> so, what did it install? ok, reading the \textit{script}
    \item<3-> Oh, nice, it's all in the \texttt{public} schema
    \item<4-> Oh, very nice, no \texttt{ALTER OPERATOR SET SCHEMA}
  \end{itemize}

  \onslide<5>
  \begin{center}
    Wait, it gets better!
  \end{center}
\end{frame}

\frame{
  \frametitle{backup and restores}

  \begin{center}
    \texttt{pg\_dump -h remote mydb | psql fresh}
  \end{center}

  \begin{itemize}
    \item<1-> extensions objects are an entire part of your database
    \item<1-> but they are maintained elsewhere, that's just a dependency
    \item<2-> \texttt{pg\_dump} makes no difference
    \item<3-> what about upgrading systems (system, database, extension)
  \end{itemize}  
}

\section{The extension specs \& scope}

\frame{
  \begin{center}
    The extension specs \& scope
  \end{center}
}

\subsection{Scope}

\begin{frame}[fragile]
  \frametitle{What problems are we solving?}

  It's all about clearing up the mess. No feature is accepted in PostgreSQL
  without complete support for dump and restore nowadays. And that's good
  news.
  \linebreak

  \onslide<2>
  \begin{example}[the goal: have pg\_dump output this]
\begin{verbatim}
CREATE EXTENSION IF NOT EXISTS hstore WITH SCHEMA public;
\end{verbatim}
  \end{example}  
\end{frame}

\subsection{Specs}

\frame{
  \begin{center}
    Specs
  \end{center}
}

\frame{
  \frametitle{How are we solving our problems?}

  Lots of little things need to happen:

  \begin{itemize}
    \item<1-> Rely on the OS to install the \textit{script} and \textit{module}
    \item<2-> Register the extension in the catalogs, to get an \textit{OID}
    \item<3-> Track dependencies at \texttt{CREATE EXTENSION} time
    \item<4-> Adapt \texttt{pg\_dump}
    \item<5-> Offer a \texttt{WITH SCHEMA} facility
    \item<6-> Offer \texttt{ALTER EXTENSION SET SCHEMA}
    \item<7-> Don't forget \texttt{DROP EXTENSION RESTRICT|CASCADE}
    \item<8-> Manage upgrading \texttt{ALTER EXTENSION UPDATE}
  \end{itemize}
}

\frame{
  \frametitle{Extensions and user data}

  What if an extension gets modified after install?

  \begin{itemize}
    \item<1-> \texttt{pg\_dump} support is all about \textit{excluding} things from dumps
    \item<1-> some extensions install default data
    \item<2-> and allow users to edit them
    \item<3-> now you want the data in your dumps, right?
  \end{itemize}  
}

\subsection{Implementation details...}

\frame{
  \begin{center}
    Implementation
  \end{center}
}

\frame{
  \frametitle{The effort in figures}

  \begin{center}
    \texttt{git diff --stat master..extension | tail -1}
    \linebreak
    \texttt{260 files changed, 4202 insertions(+), 2073 deletions(-)}
    \linebreak
    \texttt{git --no-pager diff --stat extension..upgrade | tail -1}
    \linebreak
    \texttt{125 files changed, 1976 insertions(+), 81 deletions(-)}
 \end{center}

  \begin{itemize}
    \item<2-> 5 patches, 7 branches, its own \textit{Commit Fest} section
    \item<3-> about 18 months to get an agreement on what to develop \textit{first}
    \item<4-> 2 \textit{Developer Meeting} interventions, in Ottawa, \textit{PgCon}
    \item<5-> 4 weeks full time, countless evenings, 3 months of refining
  \end{itemize}
}

\frame{
  \frametitle{What's to know, now}

  Some new commands and catalogs:

  \begin{itemize}
    \item<1-> \texttt{CREATE EXTENSION hstore SCHEMA utils;}
    \item<1-> \texttt{CREATE EXTENSION hstore VERSION 1.1;}
    \item<2-> \texttt{\char`\\dx}
    \item<3-> \texttt{ALTER EXTENSION hstore SET SCHEMA addons;}
    \item<3-> \texttt{DROP EXTENSION hstore CASCADE;}
    \item<4-> \texttt{ALTER EXTENSION hstore UPDATE TO \textit{version};}
    \item<4-> \texttt{CREATE EXTENSION hstore FROM unpackaged;}
  \end{itemize}
}

\section{Extension for their authors: YOU.}

\subsection{PGXS and the control file}

\frame{
  \begin{center}
    PGXS and the control file
  \end{center}
}

\begin{frame}[fragile]
  \frametitle{Using PGXS}

  Simpler way to have your files installed at the right place, using
  \texttt{make install}. But \texttt{Makefile}s are hard, right?
  \linebreak

  \onslide<2>
  \begin{example}[citext/Makefile]
\begin{verbatim}
MODULES = citext
EXTENSION = citext
DATA = citext--1.0.sql citext--unpackaged--1.0.sql
REGRESS = citext
\end{verbatim}
  \end{example}
\end{frame}

\begin{frame}[fragile]
  \frametitle{The control file}

  It's a very complex file containing the \textit{meta data} that PostgreSQL
  needs to know about to be able to register your \textit{extension} in its
  \textit{system catalogs}. It looks like this:
  \linebreak

  \onslide<2>
  \begin{example}[citext.control]
\begin{verbatim}
# citext extension
comment = 'data type for case-insensitive character strings'
default_version = '1.0'
module_pathname = '$libdir/citext'
relocatable = true
\end{verbatim}
  \end{example}
\end{frame}

\frame{
  \frametitle{relocatable}

  A \texttt{relocatable} extension installs all its object into the first
  schema of the \texttt{search\_path}.
  \linebreak
  It's then possible to \texttt{ALTER EXTENSION SET SCHEMA}.
}

\begin{frame}[fragile]
  \frametitle{not relocatable}

  An extension that needs to know where some of its objects are installed is
  not \texttt{relocatable}.  The extension installation script is then
  required to use the \texttt{@extschema@} \textit{placeholer} as the schema
  to work with.

  \begin{example}[tsearch2/tsearch2--unpackaged--1.0.sql]
\begin{verbatim}
ALTER EXTENSION tsearch2 ADD type @extschema@.tsvector;
ALTER EXTENSION tsearch2 ADD type @extschema@.tsquery;
\end{verbatim}
  \end{example}
\end{frame}

\begin{frame}[fragile]
  \frametitle{Extension Configuration Tables}

  \begin{example}[Flag your pg\_dump worthy objects]
\begin{verbatim}
CREATE TABLE my_config (key text, value text);
SELECT pg_catalog.pg_extension_config_dump('my_config', '');

CREATE TABLE my_config (key text, value text, standard_entry boolean);
SELECT pg_catalog.pg_extension_config_dump('my_config',
       'WHERE NOT standard_entry');

\end{verbatim}
  \end{example}  
\end{frame}

\subsection{Extensions Upgrades}

\frame{
  \begin{center}
    Extension Upgrades
  \end{center}
}


\begin{frame}[fragile]
  \frametitle{ALTER EXTENSION ... UPDATE;}

  \begin{itemize}
    \item<1-> Versions ``numbers'' are just strings
    \item<2-> Provide scripts \texttt{extension--old--new.sql}
    \item<3-> Updates only refer to changes in the SQL script
    \item<4-> Secondary control files \texttt{extension--new.control}
  \end{itemize}

\end{frame}

\begin{frame}[fragile]
  \frametitle{ALTER EXTENSION ... UPDATE [TO VERSION];}

  \begin{itemize}
    \item system view \texttt{pg\_available\_extensions}
    \item system view \texttt{pg\_available\_extension\_versions}
    \item \texttt{CREATE EXTENSION ... FORM old\_versions}
  \end{itemize}

  \begin{example}[hstore--unpackaged--1.0.sql]
\begin{verbatim}
ALTER EXTENSION hstore ADD type hstore;
ALTER EXTENSION hstore ADD function hstore_in(cstring);
ALTER EXTENSION hstore ADD function hstore_out(hstore);
...
\end{verbatim}
  \end{example}  
\end{frame}

\begin{frame}[fragile]
  \frametitle{Upgrade Scripts}

  Extension author has to provide scripts for all supported upgrades

  \begin{itemize}
    \item PostgreSQL handles upgrade paths
    \item \texttt{1.0--1.1} then \texttt{1.1--1.2}
    \item system view \texttt{pg\_available\_extension\_versions}
    \item Be careful about downgrade paths!
  \end{itemize}
  
\end{frame}


\subsection{Extensions and packaging}

\frame{
  \begin{center}
    Extension and packaging
  \end{center}
}

\begin{frame}[fragile]
  \frametitle{debian and \texttt{pg\_buildext}}

  Contributed and available in \textit{debian squeeze},
  \texttt{postgresql-server-dev-all}
  \linebreak

  \begin{example}[debian/pgversions]
\begin{verbatim}
8.4
9.0
\end{verbatim}
  \end{example}
\end{frame}

\begin{frame}[fragile]
  \frametitle{debian and \texttt{pg\_buildext}}

  Contributed and available in \textit{debian squeeze},
  \texttt{postgresql-server-dev-all}
  \linebreak

  \begin{example}[debian/rules]
\begin{verbatim}
include /usr/share/postgresql-common/pgxs_debian_control.mk

install: build
	# build all supported version
	pg_buildext build $(SRCDIR) $(TARGET) "$(CFLAGS)"

	# then install each of them
	for v in `pg_buildext supported-versions $(SRCDIR)`; do \
		dh_install -ppostgresql-$$v-pgfincore ;\
	done
\end{verbatim}
  \end{example}
\end{frame}

\section{Conclusion}

\subsection{Sponsoring}

\frame{
  \begin{center}
    Conclusion
  \end{center}
}

\frame{
  \frametitle{Money}

  4 week full time at home, thanks to \alert{2ndQuadrant}, and to our
  affiliation with European Research

  \begin{quote}
    The research leading to these results has received funding from the
    European Union's Seventh Framework Programme (FP7/2007-2013) under grant
    agreement 258862
  \end{quote}
}

\subsection{Any question?}

\frame{
  \frametitle{Any question?}

  \begin{center}
    Now is a pretty good time to ask!
  \end{center}
}

\end{document}
